% Options for packages loaded elsewhere
\PassOptionsToPackage{unicode}{hyperref}
\PassOptionsToPackage{hyphens}{url}
%
\documentclass[
]{book}
\usepackage{amsmath,amssymb}
\usepackage{lmodern}
\usepackage{iftex}
\ifPDFTeX
  \usepackage[T1]{fontenc}
  \usepackage[utf8]{inputenc}
  \usepackage{textcomp} % provide euro and other symbols
\else % if luatex or xetex
  \usepackage{unicode-math}
  \defaultfontfeatures{Scale=MatchLowercase}
  \defaultfontfeatures[\rmfamily]{Ligatures=TeX,Scale=1}
\fi
% Use upquote if available, for straight quotes in verbatim environments
\IfFileExists{upquote.sty}{\usepackage{upquote}}{}
\IfFileExists{microtype.sty}{% use microtype if available
  \usepackage[]{microtype}
  \UseMicrotypeSet[protrusion]{basicmath} % disable protrusion for tt fonts
}{}
\makeatletter
\@ifundefined{KOMAClassName}{% if non-KOMA class
  \IfFileExists{parskip.sty}{%
    \usepackage{parskip}
  }{% else
    \setlength{\parindent}{0pt}
    \setlength{\parskip}{6pt plus 2pt minus 1pt}}
}{% if KOMA class
  \KOMAoptions{parskip=half}}
\makeatother
\usepackage{xcolor}
\IfFileExists{xurl.sty}{\usepackage{xurl}}{} % add URL line breaks if available
\IfFileExists{bookmark.sty}{\usepackage{bookmark}}{\usepackage{hyperref}}
\hypersetup{
  pdftitle={自主学习指南},
  pdfauthor={宣棋},
  hidelinks,
  pdfcreator={LaTeX via pandoc}}
\urlstyle{same} % disable monospaced font for URLs
\usepackage{longtable,booktabs,array}
\usepackage{calc} % for calculating minipage widths
% Correct order of tables after \paragraph or \subparagraph
\usepackage{etoolbox}
\makeatletter
\patchcmd\longtable{\par}{\if@noskipsec\mbox{}\fi\par}{}{}
\makeatother
% Allow footnotes in longtable head/foot
\IfFileExists{footnotehyper.sty}{\usepackage{footnotehyper}}{\usepackage{footnote}}
\makesavenoteenv{longtable}
\usepackage{graphicx}
\makeatletter
\def\maxwidth{\ifdim\Gin@nat@width>\linewidth\linewidth\else\Gin@nat@width\fi}
\def\maxheight{\ifdim\Gin@nat@height>\textheight\textheight\else\Gin@nat@height\fi}
\makeatother
% Scale images if necessary, so that they will not overflow the page
% margins by default, and it is still possible to overwrite the defaults
% using explicit options in \includegraphics[width, height, ...]{}
\setkeys{Gin}{width=\maxwidth,height=\maxheight,keepaspectratio}
% Set default figure placement to htbp
\makeatletter
\def\fps@figure{htbp}
\makeatother
\setlength{\emergencystretch}{3em} % prevent overfull lines
\providecommand{\tightlist}{%
  \setlength{\itemsep}{0pt}\setlength{\parskip}{0pt}}
\setcounter{secnumdepth}{5}
\usepackage{booktabs}
\ifLuaTeX
  \usepackage{selnolig}  % disable illegal ligatures
\fi
\usepackage[]{natbib}
\bibliographystyle{plainnat}

\title{自主学习指南}
\author{宣棋}
\date{2022-03-25}

\begin{document}
\maketitle

{
\setcounter{tocdepth}{1}
\tableofcontents
}
\hypertarget{ux5e8f}{%
\chapter{序}\label{ux5e8f}}

今年上半年(2021年),我在法国一所创新学校里面做项目,临结束的时候和老板聊天,我很坦诚的提到我觉得目前的课程设计主要是还是以老师教学为主,而我个人对自主学习更感兴趣,也更希望学校里有能给孩子空间自学的课程。我的老板和我提到,他觉得自主学习本身是一种能力,也是需要教授的,而不是直接实践的内容,我参与的科学课程设计,是在不停的重复科学学习中:``观察-假设-实验-解释-结论'' 这个思维模型。当这个模型建立好了,孩子们就可以重复使用这个模型来学习他们想要学习的内容,应用在实践生活中。

这次的谈话在我的脑海中留下深刻的印象,我一直希望大家能够自主学习,也觉得学生天生拥有学习的能力,应该给予他们自主权。但其实这里面有一个关于效率的平衡点,并不是所有人都能很快的找到适合自己的路子,也可能在探索的过程中放弃,能够在混乱和复杂中摸索出自己一套模型和理论的人毕竟是少数。如果我希望推广自主学习,帮助大家拥有终身学习的能力,其实可以先帮助大家建立一个基础的模型,培养基本的能力和素养。

这是我写这一系列内容的初衷,希望给所有有意愿增强学习能力的人分享我的一点个人经验。因为语言学习是我个人的兴趣爱好,法语考出DALF C1也是我今年上半年完成了的一个学习目标,姑且以法语的学习作为应用的例子来讲授一些技巧。

\hypertarget{ux4ed8ux8d39ux5b66ux4e60}{%
\chapter{付费学习}\label{ux4ed8ux8d39ux5b66ux4e60}}

自学固然是很重要的一个能力,但我还是希望先和大家讨论一下付费学习。通过研究付费学习的项目,希望大家理解三点内容:

1,通过付费学习的课程设计,理解自学的能力框架

2,学会分析自己的情况,判断合适的学习路径

3,如果选择付费学习,如何鉴别课程

\textbf{自学是个很好的能力,但我从不排斥花钱买课,付费学习。如果有合适的课程,在我判断性价比合适的情况下,我会理所当然的选择节省时间,提升效率。因为我们最终的目的是有效率有技巧有信心的学到自己想学的内容。}

现在让我们来思考一下,当你作为学生,付费购买课程的时候,你付费购买的内容是什么?

往往来说,我们会有一个学习目标,然后去寻找能够对应这个学习目标的付费课程,如果这个课程的描述和宣传能让你有足够的信心达成这个学习目标,很多人就会选择购买。

这里,我希望大家能够拆解一下付费课程的构成,因为自主学习换句话说是自己想办法教自己,也就是说自己既要当老师又要当学生。那么,需要大家能够用老师的思维去理解这件事情。

在付费课程中,往往会提供一个学习进度表,一套准备齐全的辅助学习资料,会有能教课的老师(老师可能会有自己准备的教案),会提供伴随学习的作业。如果不是报的1对1课堂的话,也会有同学。那么如果希望自主学习达成某个学习目标,其实大致是类似的,我们需要:

学习进度安排

\begin{verbatim}
- 学习资料

- 替代老师教课的书籍或者录像或者免费网课

- 替代老师回答问题的平台

- 最好能找到一起学习的伙伴
\end{verbatim}

对应这些需求,其实我们可以分析出自主学习所需要的能力素养:

\begin{itemize}
\item
  时间管理能力

  \begin{itemize}
  \item
    项目统筹能力
  \item
    资料搜索能力
  \item
    数据分析能力
  \end{itemize}
\end{itemize}

除此之外,因为自主学习是相对缺乏监督措施以及评判措施的,它需要良好的心态和意志力要求自己执行和调整学习进度。因此自学并不是看起来很简单的一件事情,我也说如果有合适的付费项目,可以选择付费,然后凭借自己的学习能力提升学习效率。

在这个网络学习的时代,付费课程琳琅满目,那么选择合适付费课程也是一个重要的议题。这个选择包含两个部分,一个是首先搜集和整理自己的学习需求,其次是分析付费课程的的内容,看是否和自己的需求匹配。

请大家问自己以下问题:

\begin{verbatim}
- 对应此次学习项目,是否存在必须完成的考试?

- **此次的学习项目是否适合自学?**

- **我是否有足够的动力完成这项学习?**

- **我是否有富裕的时间来完成这项学习?**

- 此次学习项目的计划学习周期为多久?

- 每天/每周可分配的时间为多久?

- 学习的模式?(通勤时间的零碎学习?某个整块时间的学习)

- 我更希望自己学习还是有很多人一起互动?

- 我对完成这个目标的最大预算是多少?
\end{verbatim}

三个标粗的问题是关键性问题,只要有一个是否定答案,我都觉得更适合选择付费课程。像是乐器、高级别的运动,这些涉及到发力技巧,必须不停有人纠错和反馈的;而没有动力需要自我强迫完成的学习不适合自学,对意志力的要求太高,不建议大众尝试;时间紧张的情况下,建议也是用钱买时间,参加付费学习项目,节省自己搜集学习资料进行学习管控的精力。

剩余的问题则是指引大家选择合适的付费学习项目,特别要注意,是根据自己的时间、计划和节奏来找合适的项目,而不是反过来,被付费课程的描述吸引而选择跟课。一个课程描述写的非常美好,一个月的课程但是是每天晚上3个小时的连续课程,如果自己对应加班的时间,觉得可以咬咬牙看录播或者周末补课,这样的情况一旦落下是很难按原定计划完成,在学习意愿不够强烈的时候,都建议往少里计算和评估自己的能力,这样在完成时的满足感让人更加愉悦,形成学习的正循环。

就学习模式而言,如果希望通勤路上可以零碎的学习,就寻找有配套手机版学习小程序的课程,希望互动比较频繁和有反馈,就需要找课程有助教和交流社群的,举例如果是一位想学一门第二外语,姑且以法语为例,以上的问题答案为:

暂时没有目标考试,但是希望能简单对话,可以自学,学习动力一般(想学但没有特别强烈,想了解一下),时间富裕,学习周期三个月,每天半个到1个小时(或每周6个小时),可整可零,希望能认识些朋友,可支出预算XX元。

如果测评结果如上,然后需要找市面上的合适的课程,那么就需要找预算内法语入门类型的课程,找有助教跟踪督促学习,有同班同学交流的课程,在阅读课程介绍的时候,重点关注学习跟踪(弥补学习动力不足),有提升口语效果的(满足希望对话的目标)。

在明确自己的需求之后,可以开始课程的搜寻,特别注意一点,很多时候从自己出发,再看某些机构天花乱坠的课程介绍,会发现很多超出自己想象的添加的服务,这时候一定要稳住,把握好基本需求和额外收获的分类,避免被营销收割。

下一步找寻付费课程的时候,要通过阅读和分析课程内容来判断是否合适,内容的分析包括一般描述、课程大纲、课时安排、主讲老师等多个部分。通过描述来了解课程的学习目标,阅读大纲来了解进度,课时安排会包含课程侧重点和模式的信息,而老师的介绍可以帮助大家判断这个老师的经验和技巧。我以著名的索邦中心法语课为例,它官网的介绍非常的简短,但至少通过每周20小时中12小时课程,5小时语音实验室,3小时法国文化讲座课这样的时间表窥探教学设计。如果是我作为初学者,我肯定不会选择它,因为初学者去听文化讲座我觉得听懂的内容比例会很低,而且我希望有老师互动,不喜欢听讲座。

在看完官网的介绍之后,接下来很重要的一步是运用网络搜索相关的评价,在这个全民皆网民的时代,特别是如果瞄准的课程还是网课的情况下,在互联网的某个空间里,一定有过往学生的某些经验分享。我在此不具体描述搜索过程,但是我在浏览信息后可以得出索邦法语课''重语法,轻口语,可是相对较贵``的总结。当然,如何评价信息的真实性和可参考度又是一个话题,暂时先不在此展开了。

在课程介绍与学生评价相结合的方法后,总归会选定一门课程,我的最后一个建议就是买定离手,在评估之后决定上课,哪怕有意外和不匹配的地方,也建议坚持下来,但是把这些不匹配的点记下来以后用在之后的挑选课程中。每个人的学习方式和需求都是独特的,这里讲大概的方法,而具体的实践还是要看大家的积累。

在讲完付费学习之后,我们会正式进入自主学习的部分。培养自主学习能力过程中收获的技巧是永远都可以作为付费学习补充使用的,但如果要真的自主学习某一门学科,首先希望大家有非常强烈的学习愿望和自制力。学习驱动力这个部分,说实话外人很难帮忙,所以才会有游戏化学习、碎片学习这种抵消消极学习的方式。自制力我们还可以使用工具辅助,但是驱动力这一项还是看同学们自己。

\hypertarget{ux641cux7d22ux548cux63d0ux95ee}{%
\chapter{搜索和提问}\label{ux641cux7d22ux548cux63d0ux95ee}}

尽管自学从字面上来讲像是孤独的字眼,在事实上,自学比跟随课程学习所需要产生的互动和浏览的信息量会多很多。付费课程是雇佣老师对学习资料进行搜索,更多的技巧在于如何消化和吸收上面,这部分后面也会谈到,而这一章我想聊聊自己寻找学习资料时需要的能力,即搜索和提问。

自主学习的方式以及工具其实是在不断进化中的,我两年前用到的记笔记和时间管理的工具和现在用到的工具并不是同一个,这也是我之前懒得写作的原因之一,因为我觉得这是一个动态的存在,我的写作可能很难追的上时时的更新,而不能持久的事物,写下来终究会过时,而我又是一个深受古希腊哲学影响追求永恒的性格。我最终决定决定写作,因为哪怕只是在当下有用,相信受到启发的人也至少看到了一条路。我对自己的定位是:将找不到登山路的人引到山脚下面,然而后面的攀登还要看大家自己。我会在后文提到介绍不同的工具,但是希望大家在直接运用的同时去思考后续迭代的可能性,也尝试自己去搜索和寻找新的工具(再回来分享给我 \^{}\_\^{})

提问被我分成两类,一类是和机器打交道,即在搜索引擎,各种搜索框中的提问;另一类则是和人打交道,即和朋友、老师等被询问对象之间的交流。以我的经验,这两类提问在细节上需要遵循不同的原则,而大体的框架则是需要考虑以下两点:

\begin{itemize}
\item
  选择合适的平台/询问对象
\item
  撰写合适的重点词句/措辞
\end{itemize}

\hypertarget{ux641cux7d22ux5f15ux64ce}{%
\section{搜索引擎}\label{ux641cux7d22ux5f15ux64ce}}

我们先谈谈和机器之间的沟通。不知道大家有没有考虑过浏览器之间的区别。之前在网上看到过一个说法,进商场选购,浏览器是你脚上穿的鞋子,搜索引擎则是不同的导购。我把它延伸一下,这双鞋子质量的好坏,决定了你轻轻松松的在购物还是留下了醒目的足迹。不同的浏览器对用户隐私有着不同的政策,你平时的浏览记录可能会影响到你日常浏览时网页周边的广告。我并不是一个对浏览器研究特别深刻的人,我大概知道并且用过的有safari, chrome, firefox, tor, microsoft edge, opera.

我比较常用的safari, chrome, tor, 偶尔用firefox。

Safari对我来说最大的好处是苹果系统的设备之间云共享非常方便。手机找到一个学习资料,然而觉得需要换大屏幕(平板或电脑)观看时,用云端或者airdrop无缝转换。另外就是网站密码全都存入keychain里面,全设备共享,方便登陆不同的平台。

Chrome有一个profils,就是你打开这个浏览器,首先是可以选择账户的,我有多个个人谷歌账户,学校谷歌账户,工作谷歌账户等等..这个profils的功能帮我保持各个项目之间的独立性,每个账户有各自独立的云端硬盘(当然容量不同)、邮箱、日历、插件安装等等,提升做事情时的专注度。当自学时是同时进行不同的学科学习,或者和不同的项目相关联时,可以思考一下怎么样最方便,放在一个账户下处理,还是考虑分别建不同的账户。特别指出谷歌的插件库非常强大,能找到很多有趣的插件辅助学习。另外其实谷歌现在也可以在同账户的情况下,不同设备之间进行转换和共享,尽管我是苹果粉,但最近chrome正在取代我对safari的使用。

Tor最大的特点是保护用户隐私,以上两个浏览器都存在大家默认的收集数据和影响推送广告的情况,尽管有私密浏览的选项,然而这种切换总归是多了一步。Tor是在firefox基础上开发出的一个分支,强化了匿名的功能,在连接网络时就自动加密通信和转换地址,有一个出名的论文搜索分享网站sci-hub是被法国判定禁止使用的,用Tor的话不需要额外的方法就可以直接打开。Tor默认的搜索引擎不是google而是duckduckgo,这个在后文会提到。

Firefox介于Chrome和Tor之间,如果有朋友来访问,需要用电脑时,我都是给朋友开Firefox的隐私模式,和自己常用的分开来。

搜索引擎和浏览器有关又无关,有关是因为大部分的浏览器会默认自带的搜索引擎,比如chrome和google,Tor和duckduckgo,无关是因为,你可以选择不去用浏览器默认的搜索引擎而自己另外打开一个网站。这一段我们只讨论搜索引擎的区别。

每一个搜索引擎都有自己的排序逻辑和算法,因此首先我们要建立的第一个观念是, \textbf{使用不同的搜索引擎搜索同样的关键词所得到的的结果是不同的 。} 绝对不要被习惯给误导,只长期使用一个搜索引擎,特别是使用百度的用户,想方设法也请你尝试一下其它的搜索引擎,信息质量绝对不一样。baidu, google, duckduckgo, yahoo算是比较通用的搜索引擎吧,基本上google和duckduckgo是我选的商场导购员。这时候,我们需要建立第二个观念, ** 同一个问题换不同的语言搜索得到的结果不同 。** 使用不同语言和搜索引擎沟通的时候,尽管本质上你问的是同一个问题,然而由于资源库不同的原因,得到的答案是不一样的。就比如我想找一本好的法语词典,那么我用中文搜索''法语字典推荐`` ,英文搜索''recommendations french dictionary`` 以及法语搜索 ''le meilleur dictionnaire francais``,其实你也应该有有预感,中文的搜索结果里会显示中法词典,英文使用者会推荐一些英法词典,法语的结果很可能是法法的。第三个伴随的观念是 \textbf{不要害怕使用看不懂的语言搜索和阅读其结果,善用翻译软件。} 当我法语并没有很好,然而明确的知道我要的资料法语信息更全面时,我会使用翻译软件将我想搜索的内容转换为法语,然后使用谷歌自动翻译功能进行阅读和查找。这时候可能你使用了不同的语言搜索还是没有结果,这是为什么呢?也有可能是因为这个商场的导购只负责大类的问询,你需要进到商场里具体的门店进行搜索,而这些门店包括微信公众号/小红书/淘宝/b站/百度云盘/sci-hub等等。像是知乎quora这些用谷歌能直接搜出来的不算在内。因此第四个观念是 \textbf{可以换具体的视频/分享资源/pdf/文章平台进行有针对性的搜索。}

\hypertarget{ux641cux7d22ux5185ux5bb9}{%
\section{搜索内容}\label{ux641cux7d22ux5185ux5bb9}}

和平台同样重要的是搜索的内容和辅助搜索表达式。往往搜索需要定位关键词,在搜索的时候尽量把句子简化为词语,把连接词以及比较空泛的词删除,相反用``空格''取代,比如说``是'',``什么'',用具体的词去概括形容所希望的结果。比如说法语学习中,如果自学,可能会经常有些语法模糊的点希望能查到解释,比如说Je suis rentré chez moi. 是用etre搭配动词,而不是用的avoir,那么这个问题如果希望通过和机器交流得到答案,并不是要具体化问题``为什么是suis rentre而不是ai rentre'',相反,是要概述寻找答案,有可能你还不知道这个语法现象的名称,但是用你已经知道的概述,比如``etre 动词''的搜索结果已经可以缩小范围,之后可以进一步搜索''etre 助动词''。再举个更通俗易懂的例子,比如要问``红烧排骨怎么做'',可以搜索``红烧排骨 菜谱'' ,这两组关键词背后提问者的意图是一样的,但是搜索到的结果并不同,大家可以试试看,前者出现的视频结果比较多,后面出现的图文描述多一些,因为菜谱可以直接定位到很多和厨艺直接相关的网站。这个需要大家多多实践自己在应用中掌握到其中的技巧,一定不要怕多换几个关键词搜索。

\hypertarget{ux641cux7d22ux8868ux8fbeux5f0f}{%
\section{搜索表达式}\label{ux641cux7d22ux8868ux8fbeux5f0f}}

另外往往搜索引擎都支持搜索表达式的辅助运用,有几个比较简单的大家可以日常运用:

\begin{itemize}
\item
  ``\,'' 用双引号限定搜索内容,即搜索关键词不可被拆分,必须精确匹配,如果我想查找引用,我会加''\,'' 以免我打入的内容被分开匹配
\item
  另外,* 如果我确定一个句子的一半,那么不确定的一半可以用\emph{代替,搜索可以变形为`` } xxxxxxx''
\item
  ``-'' 减号, 例如搜索 XX - YY, 即我需要在搜索XX关键词中去掉含有YY的内容,比如说我要搜索DELF -A1,就是说忽略DELF里A1的内容 (注:DELF是法语考试,分成A1,A2,B1,B2四个级别)
\item
  ``\textbar{}'' or或者的意思,即搜索X或者Y,含有其一就能显示
\item
  ``site:'' 限定某个网站范围寻找内容,特别适用于能确定某个网站的情况,比如当我记得我在豆瓣看到过什么,我会用关键词限定豆瓣范围内搜索
\item
  ``intitile:'' 限定在网页标题里进行搜索,因为默认的是全文搜索,在目标明确的情况下限定标题可以更精准的找到内容
\item
  ``filetype:'' 限定文件类型,我一般是用来找pdf,比如''filetype:pdf'' 加书名
\item
  以上这些都可以用空格隔开的方式混合使用
  以上为给大家举的例子,如果你想知道更多的搜索方法,聪明的你已经会用浏览器配合搜索引擎和关键字``搜索技巧''来寻找信息了对不对?
\end{itemize}

除了定点搜索一步到位的方法,我需要补充的是,有的时候也不是技巧的问题,思考的层级也能决定你浏览信息的效率。比如我想找到最合适法语学习的字典,其实我可以直接搜索``法语学习经验''或者``法语学习工具'',这个时候我心里想,我希望有本字典,但是也去过来人有别的方式替代了字典的使用,或者顺便告诉我一些字典的补充工具和方法。所以有的时候搜索的时候``退一步''也可能有意想不到的收获:)

\hypertarget{ux63d0ux95eeux7684ux6280ux5de7}{%
\section{提问的技巧}\label{ux63d0ux95eeux7684ux6280ux5de7}}

最后我想讲一下和真人沟通的一些技巧。因为自主学习基本上就是靠搜索和偶尔请教一下有经验的朋友和老师,如果是付费学习,是会有机会请教助教和老师的,然而这里面遵循的原则是一样的,第一点就是 杜绝伸手党 ,也就是杜绝在不查阅任何资料的情况下直接提问,请在向人提问之前,先使用搜索引擎进行查询。

在问问题的时候,你的问题内容和你的形象是紧密联系在一起的,如果问题问的很蠢(比如谷歌搜索首页就能直接找到答案的问题),你在被询问对象那里的印象也不会好到哪里去,大家不会说出来,可能也不会立刻感觉到,但潜移默化可能会形成一个形象。我身边的朋友基本上已经被我的行为直接教会了,我觉得是信息类问题而不是需要逻辑思考和分析的问题时都会直接答''找谷哥``。

当然我们会遇到检视搜索引擎的结果仍然没有答案的情况,在搜寻未果或者答案不清楚的情况下,我给大家提供一个模板可以套用,这个模板的核心原则是 具体化 :(和搜索引擎相反)

\begin{itemize}
\item
  我在XX情况下遇到一个问题,

  这个问题是XX

  为了解决这个问题,我尝试了YYYY和已经查到YYYY

  但是ZZ仍然不太清楚,请问可以帮帮我么?谢谢!
\end{itemize}

当然视被询问对象的身份,是亲密的朋友、老师还是同学,语气和措辞可以相应变化,但是基本内容是一致的,我觉得提问时需要表达清楚背景、问题内容、已做出的努力、目前的情况和现在的疑惑。你的被询问对象一定能感觉到你的真诚和对它的尊重。这个模板同样适用于需要在论坛或者社区、频道等公共集散地使用,陌生人看到了能帮的时候也会尽量帮忙。而且这样问其实你不仅仅是在问题,你也在展现你自己解决问题的一个思路和你的思考方式,回答的人在回答的时候也会不自觉的不仅回应你的问题,如果你的思路偏了,也会告诉你为什么你找不到答案,这也能收获思维的提升。

和真人沟通,在找对人的情况下,会很高效。就是切忌不要浪费彼此的时间,如果你从一个直接的问题着手,对方还要反问很多个问题请你解释你的问题,来来回回固然最终应该也是能解决问题的,但这是对资源的一种浪费,而且如果耗时太长,也会有心累的感觉,这种情绪的负担对对话双方产生负面影响。

最后,如果你有朋友经常回答你的问题的话,要感恩呀\textasciitilde{}

\hypertarget{ux521dux6b65ux5efaux7acbux5b66ux4e60ux8d44ux6599ux5e93}{%
\chapter{初步建立学习资料库}\label{ux521dux6b65ux5efaux7acbux5b66ux4e60ux8d44ux6599ux5e93}}

注:因为这些流程需要借助一个主题将思路细节化,所以,看起来这一章接下来的内容会主要和法语学习相关,但希望大家阅读的时候不要被我提供的大量额外信息带偏,我重点想说的还是这个思路。

\hypertarget{ux4e86ux89e3ux5b66ux79d1ux6807ux51c6}{%
\section{了解学科标准}\label{ux4e86ux89e3ux5b66ux79d1ux6807ux51c6}}

现在假定,我们已经确定了一门学科想要自学,我们首先应该做什么?

首先你要去了解这个学科/行业标准,有没有哪个机构/组织设定了标准来衡量每个个体对这个学科的学习程度,标准的划分中,每个等级对应的内容是什么,有什么样的验证方式(比如考试)?对你而言,你的目标是学到什么程度?这个程度在标准里如何量化?

如果把学习当做一个旅程,以上的一系列开篇问题就是首先了解环境,同时找到自己目前的位置和接下来要去的地方,这些都确定好了,才能出发上路,并且思考哪一条路你最喜欢。

以法语为例,首先你需要了解的是《欧洲共同语言参考标准》,这个评价语言学习程度的标准适用于所有欧盟国家的语言,这里给大家放一个精简版以作参考。

\begin{quote}
\begin{quote}
\begin{quote}
A:初学阶段
A1:入门级 能理解并运用每天熟悉、与自己喜好有关且具体的表达方式和非常基础的语句,可以介绍或询问、回答自己或他人有关个人的讯息,例如居住地、人际关系、所有物,对于他人缓慢而清晰的对谈,只能以简单的方式产生反应。
A2:初级 能理解在最贴近自己的环境中经常被使用的表达方式或语句,例如非常基本的个人和家庭资料、购物、区域地理、和就业,能与人沟通简单而例行性的工作,这类工作通常只需要简单而直接的日常讯息,另外,这个等级的学习者,能够用粗浅的词语描述自身背景、以及最贴近自己的环境之中的事物。
B:独立阶段
B1:中级 能理解自己在工作、学习环境、休闲环境等等遇到熟悉的事物做出理解,能在该语言使用地区旅游时对应各种可能的状况,也可以对于自己感兴趣或熟知的的事物提出简单的相关资讯,另外还能够描述经验、事件、梦境、愿望和雄心大志,并能对自己的意见或计划做出简略的解释。
B2:中高级 能理解复杂文章段落的具体和抽象主旨,包括技巧地讨论自己专门的领域,可自然而流畅地和该语言的母语使用者进行例行互动。可以针对广泛的主题说出清晰、细节性的文字,并且可对于一个议题提出解释与利弊分析或是各式各样的想法。
C:精通阶段
C1:高级 能理解包括要求、长篇文章、或意义含蓄的广泛讯息,自然而流畅的表达,而没有明显的词穷状况发生,懂得弹性并有效率的运用语言在社交、学术、专业目的之上,对于复杂的主题能产生清晰且架构良好、细节性的文字,展现收放自如的组织形式、连结和精巧的策略。
C2:精通级 能够轻易理解任何吸收到的讯息,并且针对不同书面或口语来源做出大纲、重新架构不同的论点,提出的表达,自然而非常流畅,紧紧地抓住语言最唯妙唯肖的部分,更能在较为复杂的场合上辨别专业上细微的意涵''
\end{quote}
\end{quote}
\end{quote}

如果你对于搜索还是有些不熟练,不知道如何找这个标准,另外一个切入途径,就是直接通过你的学习目的,即自学之后想干什么的那件事儿的要求,来看对方参考了哪个标准设立门槛。反过来找,也一样能找到这个东西。一正一反两条路最后都会在中间相遇。你可以查到这个标准之后再去了解,移民需要B1,一般本科需要B2,硕士需要C1。你也可以直接去看某个项目,通过对方的要求B1/B2/C1等认识这些字母的含义。

同时,这些标准往往也对应着一些参考学时,比如法国文化协会对于法语学习给出的参考学时(累积)是:

\begin{itemize}
\tightlist
\item
  A1 60--100\\
\item
  A2 160--200\\
\item
  B1 360--400\\
\item
  B2 560--650\\
\item
  C1 810--950
\item
  C2 1,060--1,200
\end{itemize}

当你认识了这种定义性的标准,还需要去了解相关的考试,往往考试也会有好几种,而且也会面对不同的人群分组。像是法语考试,会有TCF(一般/国籍/加拿大), TEF(一般/加拿大/魁北克), DELF/DALF(一般/青少年/职业)各种,就和大家熟知的国外英语考雅思和托福(托业),国内英语有四六八级是一个道理。这里就不展开具体说这几门考试了,感兴趣的一搜就能找到介绍。

可以在网上搜索样题对不同的考试先有个概念,当然也可以把他们当做北极星,认出来就好,先放一边等到水平接近了再去研究考试。

\hypertarget{ux6574ux5408ux5b66ux4e60ux8d44ux6e90}{%
\section{整合学习资源}\label{ux6574ux5408ux5b66ux4e60ux8d44ux6e90}}

接下来,我们就要开始应有上一章写到的搜索技巧来寻找学习资源,这里有一个原则是从大到小,从整体到局部,具体一点说,就是:

\begin{itemize}
\tightlist
\item
  从课程到具体工具
\item
  从整合性资源到阶段性资源
\item
  从整体经验到针对性提高
\end{itemize}

为什么呢?因为你需要拔高自己先去山顶看一下上山的路,然后再选择,这样比直接上山会少绕圈子。如果你能找到一次性到位的完整的学习资源,那么肯定你会中间少花很多时间顾左盼右,不用一次又一次重复搜索;如果你先去读一些从0到高阶的完整经验贴,那么在这些帖子当中,一定会提到不同阶段大家使用到的工具/教材/方法,而且很多人也会分享自己的走的弯路,这些都可以整理出来进行比较。当你顺着这些方向走下去,遇到实际的问题的时候再单个解决。

这里给大家分享一些设计完整的课程:(什么课程都可以先去Edx还有coursera找找看)

\begin{enumerate}
\def\labelenumi{\arabic{enumi}.}
\tightlist
\item
  \href{https://learning.edx.org/course/course-v1:WestonHS+PFLC1x+2T2021/home}{Edx:On-Ramp to AP® French Language and Culture}
\item
  \href{https://www.coursera.org/learn/etudier-en-france}{Coursera:Étudier en France: French Intermediate course B1-B2}
\item
  \href{https://www.fun-mooc.fr/fr/cours/vivre-en-france-a1/}{Fun-mooc:Vivre en France - A1}
\item
  \href{https://www.fun-mooc.fr/fr/cours/vivre-en-france-a2/}{Fun-mooc:Vivre en France - A2}
\item
  \href{https://www.fun-mooc.fr/fr/cours/travailler-en-france-a2-b1/}{Fun-mooc:Travailler en France A2-B1}
\item
  \href{https://www.fun-mooc.fr/fr/cours/vivre-en-france-b1/}{Fun-mooc:Vivre en France - B1}
\item
  \href{https://www.fun-mooc.fr/fr/cours/preparer-et-reussir-le-delf-b2-et-le-dalf-c1/}{Fun-mooc:Préparer et Réussir le DELF B2 et le DALF C1}
\item
  \href{https://www.ecoleouverte.ca/accueil}{Ecole ouverte(québec)}
\end{enumerate}

(强烈推荐3、4、6这个系列的Fun-mooc,是法盟自己做的课程,非常棒!)

法语算是比较热门语言,网上免费的资源和课程很多,如果你学的是个小语种,或者其它小众技能,只能搜索到大家比较推荐的教材,这个时候有一个特别的技巧,原则还是我们的学习需要高屋建瓴,所以这个时候,如果你已经确认了一个很好用的教材,那么你需要的是这个教材的\textbf{教师用书}!!!

自学换一种说法就是自己教自己,你要同时扮演老师和学生双重身份,只用学生用书,你看到的内容并不是完整的,学习内容背后的知识点、内容的前后连贯性、重点需要强调的内容、各种表格总结以及建议的学习方法等等,在教师用书中都有更加详细完整的介绍。这里附赠一个网上教材+音频+教师用书的官网免费下载链接------\href{https://www.hachettefle.com/numerique/ressources}{Hachette}。

如果你居住在巴黎的话,可以免费在\href{https://bibliotheques.paris.fr/}{巴黎图书馆}办一个借书证,然后借走你需要的教材以及教师用书。如果你还不会法语的话,随便去一个现场,都会有人用英语帮助你完成,记得带护照就行。另外想学法语,有机会旅游路过的话,也都办一个,虽然是巴黎的图书馆,但和是否居住在巴黎无关,它的网站有很多免费的有声书资源,办完卡之后就有账户可以使用线上阅读和听书的功能。

网课和教材是最主流的完整性学习方案,往往一个语言的官方电台也会有分阶段的,以听力为主的学习资料,这个寻找方法在语言学习上屡试不爽(古希腊语和拉丁语除外\ldots 没有官方电台\ldots)在法国的代表网站就是: \href{https://savoirs.rfi.fr/fr}{RFI Savoirs} 和 \href{https://apprendre.tv5monde.com/fr}{TV5monde},国内有一个软件,每日法语听力,经常会收录这里面的学习材料。

最后分享一下我已经整理和使用过的法语学习app,这些app大部分如果你查找经验贴的话,应该都多多少少看到过推荐:

\begin{itemize}
\tightlist
\item
  法语词典:\href{https://www.frdic.com/}{法语助手};\href{https://www.wordreference.com/}{WordReference} (收费Antidote)
\item
  法语Dictée: \href{https://bescherelle.com/laccord-du-participe-passe}{Bescherelle}
\item
  法语影视:\href{https://www.arte.tv/fr/}{Arte.tv} ; youtube上的狗血法剧
\item
  法语泛听:Radios (app)
\item
  法语频道:youtube的个人博主们
\end{itemize}

这里我还要强调一点,就是在建立自己的资料库的时候,最开始可能会接触到各种资源和app,这个过程是一个做减法的过程,一定不能舍不得删除,你要相信,当你需要的时候,你还会再一次找到他们的,不常用的都删掉,减轻大脑里做选择的负担。

关于入门和学习资源的搜索和筛选就先分享到这里,下一章我们讨论时间管理和项目管理,当有了课程资源和工具,如何执行。

\hypertarget{ux5408ux7406ux89c4ux5212ux5b66ux4e60ux4efbux52a1}{%
\chapter{合理规划学习任务}\label{ux5408ux7406ux89c4ux5212ux5b66ux4e60ux4efbux52a1}}

当我们决定要自学,也已经找到学习资源和工具之后,并不是盲目地就开始学习,接下来有三件事情:拆解学习目标、有效执行任务、自我学习评估。

此处我想加一段关于自我状况判断,这个内容原本不在我的写作大纲里,但是最近颇有感悟,觉得应该写在这里。

\hypertarget{ux7231ux81eaux5df1}{%
\section{爱自己}\label{ux7231ux81eaux5df1}}

每个人的生活都有精力充沛的时候,也有所谓的低谷期,受过训练的人也有可能一直保持平静,我们普通人的话,这个曲线有上有下,在每个不同的值保持的时间段也不同。自学是一件需要意志力的事情,如果你决定开始自学一个内容,中间半途而废其实会影响到你的信心,经常性的达不到预期的目标,也会让人感觉沮丧,所以一开始在做计划之前,看观察一下自己的状态,判断你现在能承受怎样强度的一个学习,哪怕你意志上可以坚持,但你的身体是否有对应的能量,不会过度消耗。我不建议大家通过熬夜、挤压运动时间、快餐这种压缩时间的方式来为自学的项目腾出空间,短时间也许会看到学习效果,但这一段时间对你身体造成的伤害会在更长的将来反映出来的。

以上这段话其实有普适性,同样适用于学校的学习和职场的工作,如果到了这个程度,那说明这个项目不合适,尽快争取结束或者更换和调整。睡眠、好好吃饭、日常运动以及能让你放松的社交,对身体和精神的日常维护是一切生命活动的基础,尤其是青年时期,不要过度透支。建议大家问自己以下问题:
1、是否熬夜?能否保证高质量的足量睡眠?
2、是否有食欲?能否专心且安心的吃完每顿饭?
3、是否有力气运动?能否至少散步半小时?(或其它替代运动)

如果你每天好吃好睡好好运动,可以安排强度相对高而且时间密集的自学项目,如果状态已经不是很好,每天开心一点,从好吃好喝开始,缓缓学习就好,水滴石穿。现在很多人会把某些学习项目和机会当做救命的稻草一样抓住,认为疯狂的完成就可以改变自己的现状,我觉得要小心判断这些短期振奋人心的营销内容,救命的稻草其实是你自己的状态,你状态好的时候可以一鼓作气的冲个能快速实现的学习目标、,状态不好的时候要和自己和解,慢慢前行,也不至于直接摆烂。

另外比较常见的现象就是在学习的定目标初期,或者看到什么激动人心的学习经历分享的时候,特别容易出现打鸡血一般的亢奋期,就是头几天恨不得把所有的都学会,在短时间内很快的掌握大量的知识,废寝忘食一般的朝着目标努力。如果你的这个目标是个小目标,且确实可以在3天内达到,偶尔一次那就偶尔一次吧。时间拉长一点,在这样一个冲刺期,但凡生活内有点芝麻大小的干扰,就容易不开心,然后学习同时进入平台期时,动力就和漏气的气球一样,慢慢瘪下来了,拉长了看这个学习曲线,反而没有踏踏实实学习的同学走得远。在这个信息爆炸的时代,有太多被有目的加工过的内容,如果不能做到有效的分辨,至少要留一丝精力独立守神,不被太多的外界纷纭干扰,按照适合自己的节奏前行。

学习是个爱自己的事情,咱们不能捡芝麻丢西瓜。

\hypertarget{ux4efbux52a1ux62c6ux89e3}{%
\section{任务拆解}\label{ux4efbux52a1ux62c6ux89e3}}

吃好睡好运动好是我们进行时间分配时要守住的底线,在满足这个前提下,你应该就能估算出你可以用于学习的时间究竟有多少,然后在这个时间里,我们要开始安排具体的任务。前面我们的学习资源和工具都是按照从大到小的方式挖掘的,但是在执行学习计划,设立目标的时候,其实是相反的思路,从小到大。每一天的学习目标都应该非常具体,比如说,学语言的话,就是具体到跟读某一段音频,某一本书的某个页码,某一个语法知识点。如果你一开始完全不知道自己的学习量有多少,在前几天,你也可以反过来记录你用多少时间做了多少事情,那么这个具体的事情就是你之后制定学习计划的参考线。

可以参考其它人的学习计划,但是一定不能照搬,每个人都有自己的学习方法和学习效率,哪怕你看到一个博主它写了一个学习计划并且完成了,但实际上这个完成度有多少,中间的差值很大,而且同一项任务,不同的人完成起来也会有区别。举个例子:

看到一个博主说20分钟练习跟读了一个3分钟音频,学习了其中所有的新单词、能够流利跟读。这看起来是个很完整的描述对么?但是你知道这个博主的语言学习能力么?它是否已经有了英语基础或者西班牙语基础再学法语,新单词有多少个?和你情况完全相同么?跟读是可以中间不停的影子跟读,还是已经近似背下来了?它对语音的敏感度如何?是否在跟读过程中需要额外纠正发音?

它人的经验只能作为参考,而且你参考的部分是验证它所用资料或方法的有效性,而不是它时间的完成,这个你要通过对自己学习活动的记录来调整,直到找到自己的节奏。以自己的这个``节奏''为基本单位,进行重复性计算来算出实现阶段性目标的时间,同时在计算的时候,记得加入富余时间,比如你认为3个时间单位就可以完成,那么计划里写4-5个比较合适,这样安排的话,如果提前完成幸福感比较强,按时完成也不错,给意外的困难留出时间,也会规避计算时的误差,比如说一开始1个时间单位的定义是按照精力充沛的最好状态去设定的。

最开始的时候不要怕麻烦,对自己的学习记录要尽可能的详细,方便对自己的各项学习能力形成更完整和全面的认知,这个认知是项目管理的关键,在项目的进行过程中,也会随之继续调整,越来越符合事实。也因此这个记录除了具体完成的学习内容,如果你觉得和平时不同,比如状态特别好或者不好,都记录一下,并且写一下可能的原因,比如说饭前学习,饿到学不下去,或者邻居有噪音,影响了听力。

也因为个体的差异性,我这里很难举例子说一个具体时间对应的具体学习目标,只能随手分享一个类似参考:7:00-7:45, 在书桌前学习,晚饭散步回家,状态不错。完成了《某某教材》第X页-X页,遗留问题是\ldots.大概还有YY页要读完。通过一段时间的记录,我可以知道,我在某两项个人事务之间学习是最好的状态,比如饭后散步回家和洗漱之间,可以判断出我45分钟左右某一本教材可以学习多少页等等。

最后还有一个小tip:如果已经固定好了学习时间,建议不要因为未完成的学习任务而延长学习时间,给自己留个尾巴,第二天才好回来学习,其实人一直想看手机就是这个心理,朋友圈又更新了,这个又有新消息,总有什么在等着的感觉让你不停的拾起这些电子设备,这一点完全可以利用在学习上,不要想着一次做完,留一点才好回访啦。

\hypertarget{ux65f6ux95f4ux53caux9879ux76eeux7ba1ux7406}{%
\chapter{时间及项目管理}\label{ux65f6ux95f4ux53caux9879ux76eeux7ba1ux7406}}

在合理规划学习任务后,我们就需要时间管理和项目管理能力来有效执行任务了,也就是将自学作为一个项目,进行合理规划和进度管控,来尽可能的达成甚至超越你设置的学习目标。时间管理也可以看成项目管理的一部分,但我把它单独拿出来,是因为很多时候,大家都不是在全职学习的,那么自学和原本生活之间的关系和协调,就需要运用时间管理的技巧来通过全局掌控,毕竟如果主要的生活内容出现问题,那必然也会影响到需要运用意志力的自学项目。

\hypertarget{ux65f6ux95f4ux7ba1ux7406}{%
\section{时间管理}\label{ux65f6ux95f4ux7ba1ux7406}}

这里的时间管理包含两个层面,大的层面来讲,要对自己的日常生活有个统筹安排,区分好学习时间和其它(比如工作、学校等等),以此来保证你有富余的时间进行自己的自主学习项目,这也是上一章合理规划任务时强调的任务和生活的协调性,第二个层面是学习项目的大块时间,要如何应用,我们已经学过细化到具体的事件和学习目标,那么如果运用工具进行管控和跟踪。

我在很久之前就开始研究时间管理的各种工具和方法论,貌似2015年的时候写的一篇关于时间管理三个原则以及相关应用的分享被推荐上过主页,但留言的人也不多,我也没明白是怎么一回儿事。\href{https://www.douban.com/note/503440000/?_i=7795608Lsx5Xfa}{原文见此: 时间管理二三事} 其实过了七年,中间也有尝鲜,但大浪淘沙,常用的工具还是这些,那我也就可以放心推荐了。

对于第一个层面,生活上的统筹,看完上一章,你应该有个大体的想法了,这时候我们可以借助\href{https://www.rescuetime.com}{Rescue Time}对你的电脑应用进行跟踪,我们不想从吃饭、睡觉和运动里节省时间,但是我们可以分析出过度刷剧、浏览无效信息的时间,并加以控制和统筹。也可以借助\href{http://www.bumblebeesystems.com/wastenotime/}{Waste No Time (Extension)} 这样的工具锁定相关网址,限定浏览时间。从这里挤点时间支援一下学习项目还是好的。

在对生活分块,或者定时安排自主学习时,日历是一个基本的工具,就是苹果自带的iCalendar 或者Google Calendar,或者其它日历哪一个都行,只要它满足以下几个条件:
多平台同步、可以设置提醒、可以重复、可以加入其它邀请人、可以增加具体的描述(地点、事件内容等)。

如果你从来没用过日历来统筹自己的日历表,那可能需要花点时间将以上列出的几个功能都先练熟。之后至少在日历上明确圈出自己的自主学习时间、设置提醒,如果多次不能按时开始学习,就要跟着调整日历。前一章提到的学习前任务设定以及学习后描述,可以单独写在excel表格、或者notion的文档,或者直接写在日历的备注里。不仅是学习时间,当你习惯以后,可以在日历上统筹安排自己的生活,包括不限于工作、购物、社交等等,用不同的颜色表示,在日历上一目了然。

如果你能跟着日历上的限定时间保持专注,那么其实你不需要时间跟踪的管理软件,但是如果你容易在学习时间被带跑了,那么你可以打开\href{https://toggl.com/track/}{toggl track} 在开始学习和中断学习的时候都进行计时。哪怕你忘记了手动开始和暂停,它的后台会记录电脑上的操作,方便之后回顾。

以上的时间管理着眼于两件事情,一件事情清楚的划定合适的学习时间,另一件事情是在划定的时间内保持专注。

对于需要白噪音或者轻背景音乐学习的朋友,可以考虑配合使用\href{https://tide.fm/}{Tide},结合了番茄时钟的背景音提供工具。

最后想分享三点在实践时间管理后我理解到的事情:

第一点:不要为了表面的充实去找各种事情填满你的日历。我知道有些朋友可能会有强迫症,也可能因为焦虑如果不安排事情就觉得浪费时间,从而产生愧疚感。如果你真的看着空白的日历难受,请大胆的在日历模块填两个字:``自由''。看到这两个字多么幸福啊!到了那个时间想到做什么就去做好了,每次换也好,在自由状态下有了新的主意,固定新的安排都好,这样你的日历也是满的状态。避免安排大量消耗精力的事情在一起,或者避免报复性旅行和消费,你已经觉得累了,这时候要休息,而不是继续消耗你的精神。注意这里不是说长时间工作(静止)-\textgreater{} 暴走旅行、购物(运动),关注精神健康,身体是一个方面,要动起来,但是不要换一种方式唤醒你本该休息的内在。

第二点:时间自然填满的状态也可以是放松的!有效的时间管理并不等于紧张和压力。以前总觉得,将时间表填的满满的,一件事情接一件事情,做事就和赶集或者去打仗一样,急的不行,其实并不是这样的,将时间合理分配之后,我只是更清楚的按照功能区分了我的时间模块,学习和娱乐社交都可以是放松并且专注的,比如我和朋友在一起聊天时,也会和学习的时候一样把手机彻底静音扔到一边。而且将时间提前计划好,并不是将所有事情都装在脑子里,充满了压力,恰恰相反,是将所有安排都忘记,你可以完全信赖自己的日程单,你只要看提醒无脑做事就好了,大脑不需要额外记东西。

第三点:提前规划时间并不意味着你一定要这样执行。也许一开始你觉得这是个自相矛盾的看法,我提前规划时间就是为了执行呀,是这样没错,但是生活中充满了意外,我们要学会接受它,也许临时就要帮朋友或需要朋友帮助,有时间表会让你不会错过本应完成的事情,更方便进行调整。有了时间表之后,千万不能向机器人的方向进化,使用一切工具的原则都是让我们自己生活的更真实,如果违背了这样的原则,不如不用。(这个是谈论自己的时间安排,如果是和多人的共同安排,当然还是尽量遵守,不随意改约。)

所以我有时间规划,但有时候也未必会按照时间规划执行,但我知道哪一个执行了,哪一个还需要执行。有的时候我也不会排出一天的时间安排,而是按照我的状态直接开始做事情,用toggle记录之后,反向计入日历,这些都是可以的,大家开始的时候可以严格执行进行尝试,后期熟练了根据个人情况灵活运用\textasciitilde{}

\hypertarget{ux9879ux76eeux7ba1ux7406}{%
\section{项目管理}\label{ux9879ux76eeux7ba1ux7406}}

学习并不是一个单线程的事情,很多时候哪怕按照教材一点一点的往下学习,中间也可能突然遇上不明白的地方,或者需要查阅资料的引用内容。就语言来说的话,按照教材往下学,你读懂了不代表你能听懂,你听懂了也不代表你能写或者说出来,所以在实现一个整体性的学习目标的时候,往往都是多部分构成的,这个时候多线程之间的调控,我们可以运用一些项目管理的技巧提升效率。

这里我想参考的是GTD(get things done)的部分原则,完整的理论大家可以自行搜索。我理解到的部分就是``收集-整理-归类-回顾''这个原则。也就是当你学习中突然想当你要学习另外一部分内容,或者在读经验贴借鉴到什么经验和工具,不要立刻去做,而是可以收集到一起,比如todo清单、记事本或者GTD专门的工具等等,然后拿出一个完整的时间遍览所有你收集在一起的任务,如果几分钟内可以快速进行安排,那就立刻做,如果不能的话,就归类,比如语言学习的话,按性质归到语法、单词、听力、阅读等等这些小项目中,或者按照你认为需要推进的时间,比如本周、月底、下个月,或者按照学习程度的时间线A2,B1,B2等等,之后在你完成同类别的任务的时候加入这些新任务。回顾就是执行过后,在合适的时间点重新浏览任务,将完成的划掉,漏掉的进行重新安排。

因为自学不是等着老师投喂,而是自己不断学习不断接受新的信息的过程,所以可能经常被实时带跑,这时候还是要有一根学习的主线(课程/教材/方法),然后围绕主线安排其它细微的学习任务,在遇到更好的办法的时候可以更换主线,但是一旦更改也要稳定下来,否则会不容易找到合适的方法进行评测。

\hypertarget{ux81eaux6211ux5b66ux4e60ux8bc4ux4f30}{%
\chapter{自我学习评估}\label{ux81eaux6211ux5b66ux4e60ux8bc4ux4f30}}

\hypertarget{ux5f52ux7eb3ux4e0eux81eaux67e5}{%
\section{归纳与自查}\label{ux5f52ux7eb3ux4e0eux81eaux67e5}}

一般自学的人比较困惑的一点就是,我学了一段时间之后到了什么程度?如果有老师的话,可能老师这时候能帮助做出评估,如果是参加课程的话,一般也有配套没有老师的话,我们一般可以通过对比教材和测试的方法来进行自我探知,当然这个可能是粗略的,但是没关系,能参考就行了。

这里我们需要探讨的是频率。多久应该进行这样的自查。每次学习活动的结束都应该有一次归纳,可长可短,哪怕是半小时的学习,也可以一句话归纳出完成的内容,自查和这样的归纳还是有不同的地方,归纳是完全根据你当下的学习内容,立足点是你,而自查是你的学习和某参考标准之间的对比,立足点是比较,因此自查是在归纳好的基础上,且用的时间更多一点,它是建立在你能掌握你的学习内容的情况下。我曾经见过这样的学习情况,就是笔记写了一堆,但问学到了什么,反而是迷迷糊糊的,这样缺乏归纳总结时,更容易找不清楚自己的位置。

自查需要配合你的学习强度,如果你是高强度学习,比如每天5-8小时的话,那么每天都可以分配30分钟进行整理,如果是每天1-2小时这样,可以按照每周的频率来整理,隔日的学习可以两周一次,最低频率应该每个月有一次自查,频率过低的话我想这就不算是连续性的自学项目了,应该算作爱好培养\ldots{}

另外就是这个自查应该单独记录,当然你可以和平时的笔记写在一起,只要有单独调用的方式,能够按照时间只查看自查的内容即可。(举例,如果你使用双向链接笔记的话,就能实现既分离又一起的写作方式)这样你有专门的log记录,这样的记录,因为是建立在对比上的,内容应该尽可能用标准化的叙述,比如说,已完成A1阶段XX语法点,已学习XX类别的单词Y个左右,A1测试的听力题目5道题答对4道。就是说引用参考标准里的语句和测试来定位你的学习内容。

任何学习都会有上升期还有高原期。上升期的时候,任何一点点的学习都会看到回报,而高原期的时候,学习感觉总在绕圈了,原地踏步。因此两者在写自查记录的时候有一点小区别,上升期强调质(quality),高原期写量(quantity),因为可能上升期的每次学习都对应参考标准里的新要求,而高原期的学习,则是反复在同一个要求里努力。

以语言学习为例,高原期一般有三个:从0-A1入门,因为语音和崭新语法的关系,可能觉得比较难,然后就是B1-B2是一个很长的期限,接下来C1到接近母语这个级别要用的时间也比较长。这三个阶段,特别是B1结束到B2考证,是一般语言学习的门槛,过了B2,基本就可以无障碍日常应用了,但这个阶段需要反复的背单词、练听力、接触同级别不同的语料,这个时候在自查的时候,堆量给自己信心。

\hypertarget{ux5206ux4eab}{%
\section{分享}\label{ux5206ux4eab}}

其实我这一点做的并不够好,也在努力地改善中,也是因为这方面做的不足,所以意识到了分享的重要性。

以前我制定学习计划,并不会给分享预留时间,因为总觉得没有达到目标,而且也不确定我是不是在一条正确的路上,有没有资格能给大家分享。后来我发现,弯路也是经验,如果等到达到目标再分享,中间的过程实在冗长,已经做完事情再回头写作其实不如一边做一边写。而且当你达到目标的时候可能又开始一个新的学习计划,这个时候在有限的时间内,你是分享之前已完成的学习呢还是现在正在进行的学习?两难。
所以现在,我想比较合适的方法,应该自查和分享连在一起,自查的频率同样适用于分享,不至于太频繁而空洞没有内容,也不至于一直拖到学完。

而且前面提到的没有老师评估的问题,其实在分享中也可以被帮助。因为广大的网友和学习同伴们,有希望吸取经验的读者,也有刷到之后来回忆往昔的读者,在分享里留一两个问题,说不定就被解决了呢。而且每次分享应该都会收到鼓励,让你继续坚持下去。

分享的方式有很多种,在相关论坛/小组里持续更新帖子,在vblog的社区上传视频,和一起学习的朋友们开个共享文档,这些都可以,选择让自己舒服的方式分享就好,如果能在分享中找到共同学习、频率相当的小伙伴,会非常有帮助!我在学法语的时候,和另外两个小伙伴搭伙,特别感谢她们,一起进行口语练习。

\hypertarget{ux8c03ux6574}{%
\section{调整}\label{ux8c03ux6574}}

在归纳自查之后,并不是说我们按照这样的模式一直往前推进就可以了,而且参考项目管理
Agile方法论,要对自己的学习方法进行调整。可以通过自查的log进行归纳:哪一种学习方式最适合我?包括但不限于学习节奏,比如专注多久和休息时间的分配,学习空间和时间的选择,学习材料的喜厌、通过测试证明有效/无效的方法等等,这些都是需要自己在阶段性学习时进行分析,然后再实验新的学习方法。在分析后,也同时参考它人的经验贴或者分享后的恢复等等,可以有目的的设计实验学习模块,测试不同学习方法和自己的匹配性。重点就是一定要自己测试,不盲目跟风,找到最适合自己的方法。

``适合''在这里需要解释:一是,别人好用的不一定适用于你,原因很多,大家学习的背景基础不同,熟练应用的工具不同,所以同一个方法产生的效果不同太正常了。二是适合并不一定等同于高效,如果一个方法很高效,但学习过程让我觉得痛苦,那在我的标准下,我就会标注不适合,我会选择那个让我觉得开心但是没那么高效的方法,因为我觉得学习持久性这一点对我来说更重要。

我喜欢自主学习的一个原因就是这种灵活性,你可以根据自己的情况实时调整,而付费学习的项目往往是一个设计好的整体,必须跟着走完。但付费学习的好处就是,你花钱让别人承担了设计工作,自己不费脑子,只要完成学习任务,基本就能实现学习目标。

整个学习的方法论:自学与否的选择、建立资料库、规划学习任务、执行、评估和调整,这样就是完整的一条线了。学习这件事儿,最重要的是你去学,只要在路上了,慢慢的通过细节调整,每个人都会享受这个过程,因为你知道你在朝着更好的自己迈进。学习是生活的一部分,我们每天都在自知或者不自知的通过不同的方式学习着,希望这个指南对大家有帮助。

\hypertarget{ux8865ux51451ux8df3ux51faux8ff7ux832bux627eux5230ux65b9ux5411}{%
\chapter{补充1:跳出迷茫、找到方向}\label{ux8865ux51451ux8df3ux51faux8ff7ux832bux627eux5230ux65b9ux5411}}

自主学习指南的开篇是假定了大家已经设定了一个学习目标去实现,但是很多人会问一个问题,我没有什么想学的,怎么找到自己想学的内容呢。

首先,如果你因为这个感到痛苦的话,这个痛苦才是首先要解决的问题。这是由于对未来未知的迷茫而产生的对当下的焦虑。如果你的未来是已知的,那么你已经在朝着目标努力的路上,不会纠结你要学什么,肯定是有无穷的内容等着你。大概率是你要更换什么,但不知道去哪里,因为对未来的迷茫,导致了当下的束手无策,而且一般还有个时间线的压迫感,比如当下的这个状态在几个月之后就一定要换了什么的。这个情况下,我们最需要做的不是去寻找你接下来的路,而是调整心态,告别焦虑而且享受自由。

如果你还热爱生活,有着对未来的向往,那就是没问题的。因为学习和生活其实是一个事情,人只要活着就是在成长的,如果你不知道要学什么,但是知道你想过什么样的生活,那你就尽可能的往这个方向靠拢,然后自然而然会找到学习的方向。对于当下来讲,一个是停止焦虑、在保证身体健康遵守好好吃饭睡觉运动的前提下享受生活,另一个就是去体验一些之前没有做过的事情,参加不同类型的活动。

在参加活动和体验的时候,可以把以下几个问题放在心底:

\begin{enumerate}
\def\labelenumi{\arabic{enumi}.}
\tightlist
\item
  十年后我希望生活在哪里?
\item
  十年后我希望的日常生活是怎样的模式?
\item
  哪些职业可以满足我这样的要求?
\item
  这些职业分别对应的性格特点是哪些?核心技能是什么?
\end{enumerate}

解释如下:

第一个问题,如果你已知具体的地点那最容易,如果不知道的话,至少可以回答城市或乡村,国内或海外?海外的答案直接就涉及到外语学习,国内的话,国际大都市、卫星城、二线、乡村等等对应的消费水平和工种的选择空间差别很大,直接关系到职业的选择,乡村的话可以去学手工或者不依赖于地点的相关技能了。

第二个问题,日常的生活有很多种,有朝九晚五的,也有不需要坐班的,还有完全的自由职业,自己开店自己说了算的老板们,这些年流行起来的数字游民等等,你希望有一份工作来帮你框定日常生活的时间安排,还是希望有完全的自由自己安排,你想要常常去旅游,还是长久的在一个地方定居?根据这个问题的答案,也可以筛选职业。

第三个问题,我觉得吧,我们就不要想着一夜暴富全靠被动收入不工作来实现美好生活了,我们可以以和工作共存的方式展开想象,如果有幸碰上时代红利不用工作,那也不影响现在的思考。从最坏的可能性上做预算吧。想明白第一个第二个问题,就去列举和收集这些满足你要求的职业,然后一个一个根据自己的喜好先筛选一番。

第四个问题,假设第三个问题的答案是多解,需要做出选择,那么就列举这些工种需要的性格特点和核心技能,参照自己的情况再筛选一轮,如果还不知道,那就从共通的核心技能入手学习。注意,这里的核心技能,用前瞻性的眼光去看待它,在未来有没有可能被替代,职业技能的更新换代也是很快的。

开篇也说了这是适用于不知道自己想干啥的朋友们,已经知道想干啥的,朝着目标奔就行了,这些问题都可以放一边。
关于体验和旅游,其实我在16年写到的方法还是可以继续用,概述一下,就是说,当你去到一个地方,不要管别人的经验贴说这个一定要去还是那个一定要去,你跟着自己的想法去考虑你想去哪里,你的兴趣就藏在你的选择里。另外,在你旅游的过程中观察自己的行为,对自己的性格特质进行归纳和判断。

科技体验馆、艺术和历史博物馆、动物园、植物园这些的对应都是显而易见的,夜晚的酒吧是热爱社交,去徒步的话喜爱自然,很有耐力等等,大家可以自己联想。对自我的判断,参考之前写的内容:(更具体的内容参见日志:如何找到自己的天赋/兴趣所在)

\begin{itemize}
\tightlist
\item
  ``自己擅长一个事情,除了对一个事物拥有探究的兴趣,还应伴随将其以某种形式实现的技能或方法.这一天的活动中,你是否会合理规划路线,尽量不走回头路?(效率至上)
\item
  你是否会仔细记录花费,在付款前认真对账单,算自己的花费并对比预算?或者思考去一个地方打车合算还是公共交通合算?(对数字敏感、经济/财务类)
\item
  在游览中,你是否喜欢不停的摄影,而且摄影本身的兴趣甚至大于你在做的事情,而之后你会耐心的洗照片或者修片?(只是想拍下生活还是真正热爱摄影、艺术)
\item
  而游览结束之后,你会选择以日记或者其他方式记录并与大家分享么?(是否热爱写作、有做媒体类、编辑类的潜质)
\item
  你会由着性子随意玩还是严格执行计划呢?(自由职业还是某个大型组织中一员)
\item
  在某个地方遇到搭讪的陌生人,你会选择结伴同行还是打个招呼就别过了呢?(是否属于社交型;销售、人力资源、公共关系是否合适)
\item
  如果你遇到了骗子,你是否会义正言辞的拒绝对方的不合理需求,或者你是否愿意花时间与其争论到底?(法律、辩论)
\end{itemize}

我只是列举了某些常见的特点、行为表现方式,你会怎么表现,不妨静心问下自己,或者真的去某个安全的城市,一个人游览、四处转转,看下自己的选择的排序,从旁观者的角度观察自己的行为方式并记录下来反省自身,你会对自己的性格、能力、兴趣有更深的认识,也就对自己的想要从事的职业大方向有所思考了。''

另外还有一个有别于寻找兴趣的方法,从另一个角度切入,就是你去听别人对你的评价中赞誉最多的部分,这可能是你能产生更多最多社会价值和体现个人价值的地方,如果真的所有内容对你来说都是同一的,那么你从自己做的最好的部分入手去深化细耕变成职业也是很好的。(只要同时你觉得快乐,或者你觉得你的收获足够。)

最后,职业的从事有时候也是带有偶然性的,读一本书,听一个讲座,看一场电影,聊个天,认识个新朋友,这些生活里发生的事情都可能会影响到我们的改变,没什么大不了的,享受现在的生活,不要想着一条路走到底,所有的弯路也都是生活给予的经历和财富的一种,只要是开放的心态,不抱有那种一次买定离手的想法,一边学一边走,都可以的。

如果你的痛苦已经导致你失去了行动力,以上内容皆不适用的话,那么我觉得是要先尝试寻求专业帮助,回到正常生活的状态,如果这也很难的话,就执行一点好了:出门晒太阳,保暖的情况下在公园绿地边坐着发呆。一直晒一直晒一直晒,一直发呆一直发呆一直发呆,你就好了。相信我的话你就试试看。这个同样适用于因为任何原因导致情绪低迷的情况,前提是有太阳直接照射,阴天多云下雨都不算数哦。

希望大家都找到或大或小感兴趣的内容,早日实践自主学习\textasciitilde{}

\hypertarget{ux8865ux51452ux65e5ux5e38ux4f7fux7528ux7684ux57faux7840ux5de5ux5177}{%
\chapter{补充2:日常使用的基础工具}\label{ux8865ux51452ux65e5ux5e38ux4f7fux7528ux7684ux57faux7840ux5de5ux5177}}

最后这个补充我在想会不会太鸡肋了?还是写了,和大家分享一些我日常高频使用的学习工具,真没有什么特别的,就选了最具有普适性的。(默认有一台电脑,我个人不用平板或者手机学习,学习的时候所有有推送的软件全关闭。)

G-Suite:

虽然好像太小白了,但这就是最基础的工具包,就是Google一整套工具,你只要注册一个邮箱,就能拥有全部,包括各种在线文档编辑(doc,excel,ppt,form etc.),云盘,日历,地图,youtube, 搜索,视频会议。这一套连在一起,非常全面,基本满足日常标配的需要,而且全免费。说实话,国内的朋友们,我觉得它值得你科学上网配一套。

谷歌唯一受人诟病的就是个人隐私信息的保护,很多时候你看到的广告和你最近心里的想(实际搜索关注)的内容一致,就是它的锅。看你对这个有多在意了。

我不确定我的读者是否熟悉这些工具,我简单的总结一下为什么要用它:

在线文档编辑:永远不用担心忘记保存或者断网断电导致的数据丢失,随时可以邀请共学的小伙伴参与
搜索:虽然也有广告,但是至少比度娘靠谱很多
云盘:免费15G,存书存文字是够了
地图:分类保存常去和想去的地点,一目了然
视频会议:不需要下载,分享链接就可以进入
Youtube:我的学习资源库,不用额外注册新账户,离开Youtube我没法学习
Chrome: 用这个浏览器对多个账号分别管理,邮箱、文件互相不干扰,所以我逐渐减少了火狐和safari的使用

\href{https://justgetflux.com/}{F.lux}

保护视力,夜晚屏幕的颜色会跟着自动调节,很多人一开始用的时候可能会不适应,但是习惯就好了,这样也容易晚上不熬夜,用完电脑正常入睡。

\href{https://github.com/polywock/globalSpeed}{Global speed}

浏览器插件,看学习视频的时候,一般最多只能调到两倍速,但是老师讲话实在是太慢了,装上这个插件就可以按照自己舒服的速度随意调节了。(连广告都会一起加速哈哈啊哈)

Z-library

在海外不能回国,很多想读的书都看不了。我是愿意买电子书的,但有些书连电子版都没有\ldots 后来发现这里能找到很多一直想看但找不到途径的书,开心。不放链接了,因为会被封,链接会换,用的时候直接搜索就好。

双向链接笔记

在我参与了Roam Research的内测后,就不再使用Onenote还有notion。双向链接真的非常好用,我的法语学习笔记全都存在这里。但是它现在变成了一个收费的软件,而且很贵,导出功能也不好用,最近一直在测试替代品。

思维导图

这个软件太多了,只要能画出图来就可以用,无论是写论文整理思路,还是练习口语的时候列大纲,都很需要这种一目了然概括性的图表。要求不高的话,幕布还挺好用的。

另外自主学习指南内已经提到的时间管理工具不再赘述:
Rescue Time 全后台时间追踪
WasteNoTime 有针对性屏蔽网页
Tide 自带背景音的番茄时钟
Toggle Track 时间跟踪

以上就是我的基础包,个性化升级版含Github,Rstudio, Terminal 以及各类语言学习工具,如有必要,在单独的专题分享中再讲。

  \bibliography{book.bib,packages.bib}

\end{document}
